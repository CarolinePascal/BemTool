\documentclass[a4paper,11pt]{article}

\usepackage{a4wide,amsmath,amssymb,amsthm,color}
\usepackage[utf8]{inputenc}
\usepackage[T1]{fontenc}
\usepackage{mathrsfs}

\newcommand{\nc}{\newcommand}
\nc{\dsp}{\displaystyle}
\nc{\mrm}{\mathrm}
\nc{\mL}{\mathrm{L}}
\nc{\mH}{\mathrm{H}}
\nc{\mD}{\mathrm{D}}
\nc{\mB}{\mathrm{B}}
\nc{\R}{\mathbb{R}}
\nc{\C}{\mathbb{C}}
\nc{\lbr}{\lbrack}
\nc{\rbr}{\rbrack}
\nc{\bn}{\boldsymbol{n}}
\nc{\bx}{\boldsymbol{x}}
\nc{\by}{\boldsymbol{y}}
\nc{\bu}{\boldsymbol{u}}
\nc{\bv}{\boldsymbol{v}}

\nc{\Green}{\mathscr{G}}
\nc{\SL}{\mathrm{SL}}
\nc{\DL}{\mathrm{DL}}
\nc{\fre}{\mathfrak{e}}

\nc{\mY}{\mrm{Y}}
\nc{\vphi}{\varphi}
\nc{\vtheta}{\vartheta}
\nc{\dir}{\textsc{d}}
\nc{\neu}{\textsc{n}}


\renewcommand{\div}{\mrm{div}}
\nc{\bfX}{\mathbf{X}}
\nc{\mS}{\mrm{S}}




\date{}
\author{X.Claeys and P.Marchand}
\title{Analytic expressions of potential operators\\ in circular and spherical geometries}

\begin{document}

\maketitle



\quad\\
This document aims at giving explicit  expressions for layer potentials for various classical
equations in circular and spherical geometries. These expressions will then be used to implement
reference solutions in the boundary element library \verb?BemTool?. More precisely, these reference
solutions will be stored under the form of routines located in the folder
\verb?bemtool3/miscellaneous/? of the library.



\quad\\
For each considered case, we specify a  domain $\Omega\subset \R^{d}$ with $d=2$ or $3$, and denote
$\Gamma = \partial\Omega$ its boundary. We denote $\gamma_{\textsc{d}}:\mH^{1}(\Omega)\to \mH^{1/2}(\Gamma)$ the
interior Dirichlet trace defined by $\gamma_{\textsc{d}}(u):= u\vert_{\Gamma}$ for any $u\in \mathscr{C}^{0}(\overline{\Omega})$,
and  $\gamma_{\neu}:\mH^{1}(\Delta,\Omega)\to \mH^{-1/2}(\Gamma)$ the interior Neumann trace defined by
$\gamma_{\neu}(u) := \bn\cdot\nabla u\vert_{\Gamma} = \partial_{r}u\vert_{\Gamma}$ where $\bn$ refers to
the normal vector field ditrected toward the exterior of $\mD$. We define $\gamma_{\dir,c},\gamma_{\neu,c}$ in the
same manner, except that the traces are taken from the exterior of $\Omega$. Finally, we set
$$
\begin{array}{ll}
\{\gamma_{\dir}\}:= (\gamma_{\dir}+\gamma_{\dir,c})/2 &  \{\gamma_{\neu}\}:= (\gamma_{\neu}+\gamma_{\neu,c})/2\\[10pt]
\lbrack\gamma_{\dir}\rbr:=\gamma_{\dir}-\gamma_{\dir,c} & \lbrack\gamma_{\neu}\rbr:=\gamma_{\neu}-\gamma_{\neu,c}.
\end{array}
$$
Let us introduce the layer potentials associated to the interior of the disc $\mD$.
For any trace $v\in\mH^{+1/2}(\Gamma), p\in\mH^{-1/2}(\Gamma)$, their explicit expression
is given by:
\begin{equation}\label{PotentialOperators}
\begin{array}{l}
\dsp{ \SL(p)(\bx):=\int_{\Gamma}\Green(\bx-\by)p(\by) d\sigma(\by), }\\[10pt]
\dsp{ \DL(p)(\bx):=\int_{\Gamma}\bn(\by)\cdot(\nabla\Green)(\bx-\by)p(\by) d\sigma(\by). }
\end{array}
\end{equation}




\section{Analytic solutions in 2-D}\label{AnalyticSolution2D}


In this section $\Omega = \mD\subset \R^{2}$ is the disc of center $0$ and radius $r_{\star}>0$.
For a given point $\bx = (x_{1},x_{2})\in\R^{2}$, we shall write $(r,\theta)$ to refer to its polar
coordinates centered at $0$,
$$
\left\{\begin{array}{l}
x_{1} = r\cos\theta,\\
x_{2} = r\sin\theta.
\end{array}\right.
$$
We shall use Fourier harmonics $\fre_{n}(\theta) := \exp(\imath n\theta)$ and
write $\fre_{n}(\bx/\vert\bx\vert):=\fre_{n}(\theta)$. We will only consider equation that admit
rotational symetry (Laplace and Helmholtz equations), which will at the end of the day implies
$$
\begin{array}{l}
\dsp{ \int_{\Gamma}\overline{\fre}_{n}\cdot\gamma_{*}\DL(\fre_{p}) d\sigma = \int_{\Gamma}\overline{\fre}_{n}
\cdot\gamma_{*}\SL(\fre_{p}) d\sigma = 0 }\\ [10pt]
\textrm{for}\;*=\dir,\neu,\quad \textrm{and}\;\;n\neq p.
\end{array}
$$
We shall then deliver explicit expressions for the coefficients
$\int_{\Gamma}\overline{\fre}_{n}\cdot\{\gamma_{*}\}\DL(\fre_{n}) d\sigma$ and
$\int_{\Gamma}\overline{\fre}_{n}\cdot\{\gamma_{*}\}\SL(\fre_{n}) d\sigma$,
for arbitrary $n\in\mathbb{Z}$, and for $*=\dir,\neu$.







\subsection{Laplace equation}\label{Laplace2D}

In this paragraph, we first consider the Laplace equation $\Delta u = 0$ in $\R^{2}\setminus\Gamma$ with decay
condition at infinity. The Green kernel of this equation is $\mathscr{G}(\bx):= -(2\pi)^{-1}\ln\vert \bx\vert$.
Note that, for Laplace equation in 2D, the behaviour of layer potentials associated to the Fourier harmonics
$\fre_{n}$ with $n=0$ is somewhat special. Hence, for the sake of simplicity, we shall systematically consider
the case $n\in\mathbb{Z}\setminus\{0\}$. The layer potentials are given by
$$
\SL(\fre_{n})(\bx) =
\left\{\begin{array}{ll}
\dsp{\frac{r_{\star}}{2\vert n\vert}\Big(\frac{\vert \bx\vert}{r_{\star}}\Big)^{-\vert n\vert}\fre_{n}\Big(\frac{\bx}{\vert\bx\vert}\Big)\phantom{\imath\kappa} }
& \textrm{for}\quad \vert \bx\vert> r_{\star}\\[15pt]

\dsp{\frac{r_{\star}}{2\vert n\vert}\Big(\frac{\vert \bx\vert}{r_{\star}}\Big)^{+\vert n\vert}\fre_{n}\Big(\frac{\bx}{\vert\bx\vert}\Big)\phantom{\imath\kappa} }
& \textrm{for}\quad \vert \bx\vert< r_{\star}
\end{array}\right.
$$
and
$$
\DL(\fre_{n})(\bx) =
\left\{\begin{array}{ll}
\dsp{-\frac{1}{2}\Big(\frac{\vert \bx\vert}{r_{\star}}\Big)^{-\vert n\vert}\fre_{n}\Big(\frac{\bx}{\vert\bx\vert}\Big)\phantom{\imath\kappa} }
& \textrm{for}\quad \vert \bx\vert> r_{\star}\\[15pt]

\dsp{+\frac{1}{2}\Big(\frac{\vert \bx\vert}{r_{\star}}\Big)^{+\vert n\vert}\fre_{n}\Big(\frac{\bx}{\vert\bx\vert}\Big)\phantom{\imath\kappa} }
& \textrm{for}\quad \vert \bx\vert< r_{\star}
\end{array}\right.
$$
The boundary integral operators are given by
$$
\begin{array}{ll}
\dsp{ \int_{\Gamma}\overline{\fre}_{n}\,\{\gamma_{\dir}\}\SL(\fre_{n}) d\sigma =  \pi r_{\star}/\vert n\vert,}\\[10pt]
\dsp{ \int_{\Gamma}\overline{\fre}_{n}\,\{\gamma_{\neu}\}\DL(\fre_{n}) d\sigma =  \pi\vert n\vert/r_{\star},}\\[10pt]
\dsp{ \int_{\Gamma}\overline{\fre}_{n}\,\{\gamma_{\dir}\}\DL(\fre_{n}) d\sigma =  0,}\\[10pt]
\dsp{ \int_{\Gamma}\overline{\fre}_{n}\,\{\gamma_{\neu}\}\SL(\fre_{n}) d\sigma =  0.}\\

\end{array}
$$





\subsection{Helmholtz equation}\label{Helmholtz2D}

Here we consider an Helmholtz equation   $-\Delta u -\kappa^{2}u = 0$ with outgoing radiation condition.
The corresponding Green kernel is $\Green(\bx) = \frac{\imath}{4}H_{0}^{(1)}(\kappa\vert \bx\vert)$, where
$H_{0}^{(1)}(z)$ refers to the Hankel function of order zero and of the first kind, see \S 10.2 and Formula 10.4.3 in
\cite{MR2723248}. The layer potentials admit the expressions
$$
\SL(\fre_{n})(\bx) =
\left\{\begin{array}{ll}
\dsp{\imath r_{\star}\frac{\pi}{2}J_{\vert n\vert}(\kappa r_{\star}) H^{(1)}_{\vert n\vert}(\kappa \vert \bx \vert )
\fre_{n}\Big(\frac{\bx}{\vert\bx\vert}\Big)\phantom{\imath\kappa} }  & \textrm{for}\quad \vert \bx\vert> r_{\star}\\[15pt]
\dsp{\imath r_{\star}\frac{\pi}{2}H^{(1)}_{\vert n\vert}(\kappa r_{\star}) J_{\vert n\vert}(\kappa \vert \bx \vert )
\fre_{n}\Big(\frac{\bx}{\vert\bx\vert}\Big)\phantom{\imath\kappa} }
& \textrm{for}\quad \vert \bx\vert<r_{\star}
\end{array}\right.
$$
and
$$
\DL(\fre_{n})(\bx) =
\left\{\begin{array}{ll}
\dsp{ -\imath\kappa r_{\star}\frac{\pi}{2}H^{(1)'}_{\vert n\vert}(\kappa r_{\star})J_{\vert n\vert}(\kappa \vert \bx \vert )
\fre_{n}\Big(\frac{\bx}{\vert\bx\vert}\Big) }
& \textrm{for}\quad \vert \bx\vert< r_{\star}\\[15pt]
\dsp{ -\imath\kappa r_{\star} \frac{\pi}{2}J_{\vert n\vert}'(\kappa r_{\star} )H^{(1)}_{\vert n\vert}(\kappa \vert \bx \vert)
\fre_{n}\Big(\frac{\bx}{\vert\bx\vert}\Big) }
& \textrm{for}\quad \vert \bx\vert>r_{\star}
\end{array}\right.
$$
The boundary integral operators are given by
$$
\begin{array}{ll}
\dsp{ \int_{\Gamma}\overline{\fre}_{n}\{\gamma_{\dir}\}\SL(\fre_{n}) d\sigma
= \imath r_{\star}^2 \pi^{2} H^{(1)}_{\vert n\vert}(\kappa r_{\star} ) J_{\vert n\vert}(\kappa r_{\star}) }\\[10pt]
\dsp{ \int_{\Gamma}\overline{\fre}_{n}\{\gamma_{\neu}\}\DL(\fre_{n}) d\sigma
= -\imath \kappa^{2} r_{\star}^2 \pi^{2} H^{(1)'}_{\vert n\vert}(\kappa r_{\star}) J_{\vert n\vert}'(\kappa r_{\star}) }\\[10pt]
\dsp{ \int_{\Gamma}\overline{\fre}_{n}\{\gamma_{\neu}\}\SL(\fre_{n}) d\sigma = +r_{\star}^2 \imath\kappa\frac{\pi^{2}}{2}
\big(\; H^{(1)}_{\vert n\vert}(\kappa r_{\star} )J_{\vert n\vert}'(\kappa r_{\star} )+H^{(1)'}_{\vert n\vert}(\kappa r_{\star} )J_{\vert n\vert}(\kappa r_{\star})\; \big) }\\[10pt]
\dsp{ \int_{\Gamma}\overline{\fre}_{n}\{\gamma_{\dir}\}\DL(\fre_{n}) d\sigma = -\imath\kappa r_{\star}^2 \frac{\pi^{2}}{2}
\big(\; H^{(1)}_{\vert n\vert}(\kappa r_{\star})J_{\vert n\vert}'(\kappa r_{\star})+H^{(1)'}_{\vert n\vert}(\kappa r_{\star})J_{\vert n\vert}(\kappa r_{\star})\; \big) }\\[10pt]

\end{array}
$$

\subsection{Modified Helmholtz equation}\label{ModifiedHelmholtz2D}

Here we consider a Modified Helmholtz equation   $-\Delta u +\kappa^{2}u = 0$ with outgoing radiation condition.
The corresponding Green kernel is $\Green(\bx) = \frac{1}{2 \pi }K_{0}(\kappa\vert \bx\vert)$, where
$K_{0}(z)$ refers to the Modified Bessel function of order zero and of the second kind, see \S 10.25
\cite{MR2723248}. We also denote $I_0$, the Modified Bessel function of order zero and of the first kind. The layer potentials admit the expressions
$$
\SL(\fre_{n})(\bx) =
\left\{\begin{array}{ll}
\dsp{r_{\star} K_{\vert n\vert}(\kappa \vert \bx \vert ) I_{\vert n\vert}(\kappa  r_{\star})
\fre_{n}\Big(\frac{\bx}{\vert\bx\vert}\Big)\phantom{\imath\kappa} }  & \textrm{for}\quad \vert \bx\vert> r_{\star}\\[15pt]
\dsp{r_{\star} K_{\vert n\vert}(\kappa r_{\star} ) I_{\vert n\vert}(\kappa  \vert \bx \vert)
\fre_{n}\Big(\frac{\bx}{\vert\bx\vert}\Big)\phantom{\imath\kappa} }
& \textrm{for}\quad \vert \bx\vert<r_{\star}
\end{array}\right.
$$
and
$$
\DL(\fre_{n})(\bx) =
\left\{\begin{array}{ll}
\dsp{ -\kappa r_{\star} I'_{\vert n\vert}(\kappa r_{\star})K_{\vert n\vert}(\kappa \vert \bx \vert )
\fre_{n}\Big(\frac{\bx}{\vert\bx\vert}\Big) }
& \textrm{for}\quad \vert \bx\vert< r_{\star}\\[15pt]
\dsp{ -\kappa r_{\star} K_{\vert n\vert}'(\kappa r_{\star} )I_{\vert n\vert}(\kappa \vert \bx \vert)
\fre_{n}\Big(\frac{\bx}{\vert\bx\vert}\Big) }
& \textrm{for}\quad \vert \bx\vert>r_{\star}
\end{array}\right.
$$
The boundary integral operators are given by
$$
\begin{array}{ll}
\dsp{ \int_{\Gamma}\overline{\fre}_{n}\{\gamma_{\dir}\}\SL(\fre_{n}) d\sigma
= 2 \pi r_{\star}^2 K_{\vert n \vert}(\kappa r_{\star}) I_{\vert n\vert}(\kappa r_{\star})}\\[10pt]
\dsp{ \int_{\Gamma}\overline{\fre}_{n}\{\gamma_{\neu}\}\DL(\fre_{n}) d\sigma
= -2 \kappa^{2} r_{\star}^2 \pi I'_{\vert n\vert}(\kappa r_{\star}) K_{\vert n\vert}'(\kappa r_{\star}) }\\[10pt]
\dsp{ \int_{\Gamma}\overline{\fre}_{n}\{\gamma_{\neu}\}\SL(\fre_{n}) d\sigma = +\pi r_{\star}^2 \kappa
\big(\; K_{\vert n\vert}'(\kappa r_{\star} )I_{\vert n\vert}(\kappa r_{\star} )+K_{\vert n\vert}(\kappa r_{\star} )I_{\vert n\vert}'(\kappa r_{\star})\; \big) }\\[10pt]
\dsp{ \int_{\Gamma}\overline{\fre}_{n}\{\gamma_{\dir}\}\DL(\fre_{n}) d\sigma = -\kappa r_{\star}^2 \pi
\big(\; I'_{\vert n\vert}(\kappa r_{\star})K_{\vert n\vert}(\kappa r_{\star})+K'_{\vert n\vert}(\kappa r_{\star})
I_{\vert n\vert}(\kappa r_{\star})\; \big) }\\[10pt]

\end{array}
$$
Notice that we can check the previous relations formally, using the relation obtained for the Helmholtz equation with the transformation $\kappa \rightarrow \imath \kappa$, and the following relations for $x\in \mathbb{R}$ (see Formula 10.27.6 and 10.27.8 in
\cite{MR2723248}):
$$
\begin{array}{ll}
\dsp{ K_{\vert n \vert}(x)=\frac{\pi}{2}i^{\vert n \vert +1} H^{(1)}_{\vert n\vert}(\imath x)}\\[10pt]
\dsp{I_{\vert n \vert}(x)=i^{-\vert n \vert}J_{\vert n \vert}(\imath x)  }\\[10pt]
\end{array}
$$


\section{Analytic solutions in 3-D}\label{AnalyticSolution3D}

In this section $\Omega = \mB\subset \R^{3}$ is the ball of center $0$ and radius $\rho_{\star}>0$.
For a given point $\bx = (x_{1},x_{2},x_{3})\in\R^{3}$, we shall write $(\rho,\theta,\varphi)\in \R_{+}
\times \lbr0,\pi\rbr\times \lbr 0, 2\pi\lbr$ to refer to its polar
coordinates centered at $0$,
$$
\left\{\begin{array}{l}
x_{1} = \rho\sin\theta \cos\varphi,\\
x_{2} = \rho\sin\theta \sin\varphi,\\
x_{3} = \rho\cos\theta.
\end{array}\right.
$$
To conduct separation of variables, we use the so-called spherical harmonics $\mY_{l}^{m}(\theta,\vphi)$
and sometimes write $\mY_{l}^{m}(\bx/\vert\bx\vert):=\mY_{l}^{m}(\theta,\vphi)$. Here the indices $l,m$ have to
satisfy $0\leq \vert m\vert\leq l$. For the definition of these functions, we use the convention of the
\verb?boost::math? library, namely
$$
\dsp{ \mrm{Y}_{l}^{m}(\theta,\varphi):= \sqrt{\frac{(l+1/2)}{2\pi}\, \frac{(l-\vert m\vert)!}{(l+\vert m\vert)!}   }\;
\mrm{P}_{l}^{m}(\cos\theta) \exp(\imath m\varphi)  }.
$$
Here the $\mrm{P}_{l}^{m}(z)$ are the associated Legendre functions. With thsis definition, the family
$(\mY_{l}^{m})$ forms an orthonormal basis of $\mL^{2}(\mrm{S}^{2})$ where $\mrm{S}^{2}$ is the unit sphere.
Here again, we consider only rotation invariant equations (Laplace, Helmholtz and Maxwell),
so that the layer potentials will be diagonalised by the spherical harmonics
$$
\begin{array}{l}
\dsp{ \int_{\Gamma}\overline{\mY}_{l}^{m}\cdot\gamma_{*}\DL(\mY_{p}^{q}) d\sigma =
\int_{\Gamma}\overline{\mY}_{l}^{m}\cdot\gamma_{*}\SL(\mY_{p}^{q}) d\sigma = 0 }\\ [10pt]
\textrm{for}\;*=\dir,\neu,\quad \textrm{and}\;\;(l,m)\neq (p,q).
\end{array}
$$




\subsection{Laplace equation}\label{Laplace3D}

In this paragraph, we first consider the Laplace equation $\Delta u = 0$ in $\R^{3}\setminus\Gamma$ with decay
condition at infinity. The Green kernel of this equation is $\mathscr{G}(\bx):= 1/(4\pi\vert\bx\vert)$.
The layer potentials are given by
$$
\SL(\mY_{l}^{m})(\bx) =
\left\{\begin{array}{ll}
\dsp{\frac{\rho_{\star}}{2l+1}\Big(\frac{\vert \bx\vert}{\rho_{\star}}\Big)^{-(l+1)}\mY_{l}^{m}\Big(\frac{\bx}{\vert\bx\vert}\Big)\phantom{\imath\kappa} }
& \textrm{for}\quad \vert \bx\vert> \rho_{\star}\\[15pt]

\dsp{\frac{\rho_{\star}}{2l+1}\Big(\frac{\vert \bx\vert}{\rho_{\star}}\Big)^{+l}\mY_{l}^{m}\Big(\frac{\bx}{\vert\bx\vert}\Big)\phantom{\imath\kappa} }
& \textrm{for}\quad \vert \bx\vert< \rho_{\star}
\end{array}\right.
$$
and
$$
\DL(\mY_{l}^{m})(\bx) =
\left\{\begin{array}{ll}
\dsp{-\frac{l}{2l+1}\Big(\frac{\vert \bx\vert}{\rho_{\star}}\Big)^{-(l+1)}\mY_{l}^{m}\Big(\frac{\bx}{\vert\bx\vert}\Big)\phantom{\imath\kappa} }
& \textrm{for}\quad \vert \bx\vert> \rho_{\star}\\[15pt]

\dsp{+\frac{l+1}{2l+1}\Big(\frac{\vert \bx\vert}{\rho_{\star}}\Big)^{+l}\mY_{l}^{m}\Big(\frac{\bx}{\vert\bx\vert}\Big)\phantom{\imath\kappa} }
& \textrm{for}\quad \vert \bx\vert< \rho_{\star}
\end{array}\right.
$$
The boundary integral operators are given by
$$
\begin{array}{ll}
\dsp{ \int_{\Gamma}\overline{\mY}_{l}^{m}\{\gamma_{\dir}\}\SL(\mY_{l}^{m}) d\sigma
= \frac{\rho_{\star}}{2l+1} }\\[10pt]
\dsp{ \int_{\Gamma}\overline{\mY}_{l}^{m}\{\gamma_{\neu}\}\DL(\mY_{l}^{m}) d\sigma
=  \frac{l(l+1)}{2l+1}\frac{1}{\rho_{\star}}   }\\[10pt]
\dsp{ \int_{\Gamma}\overline{\mY}_{l}^{m}\{\gamma_{\neu}\}\SL(\mY_{l}^{m}) d\sigma
=  -\frac{1}{2}\frac{1}{2l+1} }\\[10pt]
\dsp{ \int_{\Gamma}\overline{\mY}_{l}^{m}\{\gamma_{\dir}\}\DL(\mY_{l}^{m}) d\sigma
=  +\frac{1}{2}\frac{1}{2l+1} }\\[10pt]

\end{array}
$$




\subsection{Helmholtz equation}\label{Helmholtz3D}

Here we consider Helmholtz equation $-\Delta u-\kappa^{2} u =0$ with wave number $\kappa>0$. The outgoing Green kernel
is given here by $\mathscr{G}_{\kappa}(\bx):=\exp(\imath\kappa\vert\bx\vert)/(4\pi\vert \bx\vert)$.
Define the spherical Bessel functions $j_{l}(z) := \sqrt{\pi/(2z)}J_{l+1/2}(z)$ and the spherical Hankel functions
$h_{l}^{(1)}(z):= \sqrt{\pi/(2z)}H^{(1)}_{l+1/2}(z)$ like in \S 10.47 of \cite{MR2723248}.
The single layer and double layer potentials admit explicit expressions in terms of the spherical harmonics.
On the one hand\\
$$
\SL(\mrm{Y}_{l}^{m})(\bx) =
\left\{\begin{array}{ll}
\dsp{+\imath\kappa\rho_{\star}^{2}\, h^{(1)}_{l}(\kappa\rho_{\star})j_{l}(\kappa\vert \bx\vert)\mrm{Y}_{l}^{m}\Big(\frac{\bx}{\vert\bx\vert}\Big)} & \textrm{for}\;\;\vert \bx\vert<\rho_{\star}\\[10pt]
\dsp{+\imath\kappa\rho_{\star}^{2}\, j_{l}(\kappa\rho_{\star})h^{(1)}_{l}(\kappa\vert \bx\vert)\mrm{Y}_{l}^{m}\Big(\frac{\bx}{\vert\bx\vert}\Big)   } & \textrm{for}\;\;\vert\bx\vert>\rho_{\star}.
\end{array}\right.
$$
and
$$
\DL(\mrm{Y}_{l}^{m})(\bx) =
\left\{\begin{array}{ll}
\dsp{-\imath\kappa^{2}\rho_{\star}^{2}\, j_{l}(\kappa\vert \bx\vert)h^{(1)'}_{l}(\kappa\rho_{\star})\mrm{Y}_{l}^{m}\Big(\frac{\bx}{\vert\bx\vert}\Big) }
& \textrm{for}\;\;\vert\bx\vert<\rho_{\star}\\[10pt]
\dsp{-\imath\kappa^{2}\rho_{\star}^{2}\, h^{(1)}_{l}(\kappa\vert \bx\vert)j_{l}'(\kappa\rho_{\star})\mrm{Y}_{l}^{m}\Big(\frac{\bx}{\vert\bx\vert}\Big) }
& \textrm{for}\;\;\vert\bx\vert>\rho_{\star}.
\end{array}\right.
$$
The boundary integral operators are given by
$$
\begin{array}{l}
\dsp{ \int_{\Gamma}\overline{\mrm{Y}}_{l}^{m}\{\gamma_{\dir}\}\SL(\mrm{Y}_{l}^{m}) d\sigma = +\imath\kappa \rho_{\star}^{2}\,j_{l}(\kappa\rho_{\star})h_{l}^{(1)}(\kappa\rho_{\star}), }\\[10pt]
\dsp{ \int_{\Gamma}\overline{\mrm{Y}}_{l}^{m}\{\gamma_{\neu}\}\DL(\mrm{Y}_{l}^{m}) d\sigma = -\imath\kappa^{3}\rho_{\star}^{2} \,j_{l}'(\kappa\rho_{\star})h_{l}^{(1)'}(\kappa\rho_{\star}), }\\[10pt]
\dsp{ \int_{\Gamma}\overline{\mrm{Y}}_{l}^{m}\{\gamma_{\neu}\}\SL(\mrm{Y}_{l}^{m}) d\sigma = +\imath\frac{\kappa^{2}}{2}\rho_{\star}^{2}
\big( \,j_{l}'(\kappa\rho_{\star})h_{l}^{(1)}(\kappa\rho_{\star}) +  j_{l}(\kappa\rho_{\star})h_{l}^{(1)'}(\kappa\rho_{\star})\,\big), }\\[10pt]
\dsp{ \int_{\Gamma}\overline{\mrm{Y}}_{l}^{m}\{\gamma_{\dir}\}\DL(\mrm{Y}_{l}^{m}) d\sigma = -\imath\frac{\kappa^{2}}{2}\rho_{\star}^{2}
\big( \,j_{l}'(\kappa\rho_{\star})h_{l}^{(1)}(\kappa\rho_{\star}) +  j_{l}(\kappa\rho_{\star})h_{l}^{(1)'}(\kappa\rho_{\star})\,\big). }
\end{array}
$$

\section{Analytic solutions to Maxwell's equations}

In the present section we use once again the notations of Section \ref{AnalyticSolution3D}, in particular concerning
spherical coordinates and spherical harmonics.  We need to introduce vector counterpart to spherical harmonics.
Considering  the $\mrm{Y}_{l}^{m} = \mrm{Y}_{l}^{m}(\vtheta)$ as functions on the unit sphere $\vtheta\in \mrm{S}^{2}\subset \R^{3}$,
we set
$$
\bfX_{l,m}^{+} = \frac{1}{\sqrt{l(l+1)}}\nabla_{\mrm{S}^{2}}\mY_{l}^{m}\quad,\textrm{and},\quad
\bfX_{l,m}^{-} = \frac{1}{\sqrt{l(l+1)}}\bn_{\mS^{2}}\times\nabla_{\mS^{2}}\mY_{l}^{m}
$$
where $\bn_{\mS^{2}}$ refers to the unit normal vector on the unit sphere directed toward the exterior of the unit ball.
The set $\{\bfX_{l,m}^{\pm}\}$ yields an orthonormal basis of the space os square integrable
tangential vector fields over $\mS^{2}$.

\quad\\
We only yield explicit expressions for the operators Electric Field Integral Equation (EFIE) and
Magnetic Field Integral Equation (MFIE). In a general geometrical setting, these are defined by the
following variationnal forms
$$
\begin{array}{l}
\dsp{ \langle \mrm{EFIE}_{\kappa}(\bu),\bv\rangle := \int_{\Gamma\times \Gamma} \mathscr{G}_{\kappa}(\bx-\by)
\big(\; \bu(\bx)\cdot\bv(\by) -\kappa^{-2}\div_{\Gamma}\bu(\bx) \div_{\Gamma}\bv(\by)\; \big) d\sigma(\bx,\by)}\\[10pt]
\dsp{ \langle \mrm{MFIE}_{\kappa}(\bu),\bv\rangle := \int_{\Gamma\times\Gamma}(\nabla \mathscr{G}_{\kappa})(\bx-\by)\cdot
(\bv(\by)\times\bu(\bx)) \;d\sigma(\bx,\by) }\\[20pt]
\dsp{\mathscr{G}_{\kappa}(\bx):=\exp(\imath\kappa\vert \bx\vert)/(4\pi \vert\bx\vert) }
\end{array}
$$
In the case  $\Gamma = \mS^{2}$ we have:
$$
\begin{array}{l}
 \langle \mrm{EFIE}_{\kappa}(\bfX_{l,m}^{+}),\bfX_{l,m}^{+}\rangle
 = \langle \mrm{EFIE}_{\kappa}(\bfX_{l,m}^{-}),\bfX_{l,m}^{-}\rangle \\
\textcolor{white}{ \langle \mrm{EFIE}_{\kappa}(\bfX_{l,m}^{+}),\bfX_{l,m}^{+}\rangle }
 = (\imath/\kappa)(\;j_{l}(\kappa)+\kappa j_{l}'(\kappa)\;)(\;h^{(1)}_{l}(\kappa)+\kappa h^{(1)\prime}_{l}(\kappa)\;)  \\[10pt]
\langle \mrm{MFIE}_{\kappa}(\bfX_{l,m}^{+}),\bfX_{l,m}^{-}\rangle =
\langle \mrm{MFIE}_{\kappa}(\bfX_{l,m}^{-}),\bfX_{l,m}^{+}\rangle \\
\textcolor{white}{\langle \mrm{MFIE}_{\kappa}(\bfX_{l,m}^{+}),\bfX_{l,m}^{-}\rangle}
=  -\imath \lbr j_{l}(\kappa)(h_{l}^{(1)}(\kappa)+\kappa h_{l}^{(1)\prime}(\kappa)) +
h_{l}^{(1)}(\kappa)(j_{l}(\kappa)+\kappa j_{l}'(\kappa)) \rbr\\[10pt]


 \langle \mrm{EFIE}_{\kappa}(\bfX_{l,m}^{+}),\bfX_{l,m}^{-}\rangle  =  \langle \mrm{EFIE}_{\kappa}(\bfX_{l,m}^{-}),\bfX_{l,m}^{+}\rangle  = 0\\[10pt]
 \langle \mrm{MFIE}_{\kappa}(\bfX_{l,m}^{+}),\bfX_{l,m}^{+}\rangle  =  \langle \mrm{MFIE}_{\kappa}(\bfX_{l,m}^{-}),\bfX_{l,m}^{-}\rangle  = 0

\end{array}
$$






\bibliography{biblio}
\bibliographystyle{plain}
\end{document}
